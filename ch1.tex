\chapter{Introducción}

	\section{Estado del arte}
	
	Una de las fundamentales a entender cuando se habla de detectores infrarojos es que \cite{Rogalski2020} (pag 34)
	
	Una de las fundamentales a entender cuando se habla de detectores infrarojos es que \cite{Rogalski2020} (pag 60)
	
	Esta es otra cita \cite{He2000}	
	
	
    \lipsum[1-3]


    \section{Objetivos}
	
		\subsection{Objetivo general}
			\begin{itemize}
				\item Diseñar e implementar en FPGA un TRNG híbrido para la generación de secuencias muy largas.
			\end{itemize}
		
		\subsection{Objetivos específicos}
			\begin{itemize}
                \item Investigar el estado del arte de diferentes generadores de números aleatorios.
                \item Estudiar los diferentes tipos de generadores de números aleatorios y analizar sus características principales.
                \item Estudiar la teoría de los mapas caóticos y su utilidad en generadores de números aleatorios.
                \item Diseñar un generador de números aleatorios híbrido utilizando un TRNG como generador de semillas y un mapa caótico para realizar un postprocesamiento que mejore sus características estadísticas y comprobar estas utilizando las pruebas NIST.
                \item Implementar el TRNG híbrido en una FPGA.
                %\item Utilizar las pruebas NIST para comprobar sus característica estadísticas.
			\end{itemize}
