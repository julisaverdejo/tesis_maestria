\chapter{Conclusiones}

\begin{itemize}
 \item Uno de los principales logros de este trabajo fue el diseño de códigos robustos, modulares y escalables para la implementación del protocolo SPI en Verilog, específicamente para el control de los convertidores analógico-digital (A/D) y digital-analógico (D/A), ADS7841 y MCP4922, respectivamente. La robustez del SPI fue comprobada a través de simulaciones, mediciones con osciloscopio y multímetro, así como por la recepción de datos mediante MATLAB. Estos códigos no solo cumplen con los requisitos del proyecto, sino que además tienen la flexibilidad de ser reutilizados y adaptados a futuras necesidades. Su diseño modular permite modificaciones fáciles y rápidas.
 \item Se logró diseñar una PCB funcional con una matriz de 8x8 fototransistores, capaz de detectar las variaciones de luz. La PCB no solo demostró ser eficiente en su propósito, sino que también cumplió con todos los requisitos de diseño y fabricación, lo que permitió que fuera producida y ensamblada exitosamente. Este resultado valida tanto el diseño como su implementación práctica, asegurando su utilidad en aplicaciones futuras que requieran la captura de imágenes a través de variaciones lumínicas.  
 \item El módulo de caracterización desarrollado no solo demostró ser eficaz para las pruebas realizadas, sino que también ofrece la flexibilidad necesaria para ser modificado y adaptado para la caracterización de un microbolómetro real. Este enfoque modular permite que el sistema se ajuste a diferentes requisitos y aplicaciones.
 \item Aunque en este trabajo el uso de una resistencia de referencia y un capacitor fueron suficientes para emplearse como circuito de lectura, no significa que estos vayan a servir para un cualquier tipo de sensor. Es necesario contar con un circuito de lectura específico adaptado a la señal de salida generada por el detector en uso. Esto es esencial para asegurar una captura precisa y eficiente de los datos.
 \item Para obtener lecturas adecuadas de una matriz de detectores, no es suficiente contar únicamente con convertidores A/D y D/A de alta resolución. Es igualmente crucial utilizar multiplexores que no introduzcan una resistencia excesiva en el circuito. La resistencia añadida por los multiplexores altera las mediciones, afectando la precisión del sistema de adquisición de datos. Por lo tanto, es esencial seleccionar cuidadosamente los componentes para minimizar estas interferencias y obtener lecturas precisas y confiables.
\end{itemize}


    
