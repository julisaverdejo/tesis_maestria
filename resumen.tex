\chapter{Resumen}

En este trabajo se diseñó e implementó un sistema de adquisición de datos para obtener imágenes a partir de un arreglo de fototransistores que emulan el comportamiento de un microbolómetro. Se utilizó una FPGA Artix 7 para controlar y sincronizar todos los módulos necesarios para la adquisición de datos, entre los cuales se incluyen un módulo ADC, un módulo DAC, dos módulos multiplexores y una PCB con la matriz de fototransistores. Para cumplir con este objetivo, se desarrolló un protocolo SPI personalizado que controla los convertidores analógico-digital (A/D) y digital-analógico (D/A). Además, se llevó a cabo la caracterización de resistencias y multiplexores mediante un circuito de lectura, lo cual incluyó un barrido de voltaje para evaluar las características I/V. También se caracterizó una fotorresistencia bajo diferentes condiciones de iluminación, utilizando un PWM con diversos duty cycles. Finalmente, se diseñó una PCB para la matriz de fototransistores, destinada a obtener imágenes a través de la detección de variaciones lumínicas.