\chapter{Resumen}

En este trabajo se diseñó e implementó un sistema de adquisición de datos para la obtención de imágenes provenientes de un arreglo de 8x8 fototransistores que emulan el comportamiento de un microbolómetro. Como herramientas se utilizaron una FPGA Artix 7 para realizar el control y sincronización de todos los módulos, entre los cuales se incluyen: un módulo ADC, un módulo DAC, dos módulos multiplexores y una PCB con la matriz de fototransistores. Para lograr este objetivo, se diseñó un protocolo SPI personalizado para controlar los convertidores analógico-digital (A/D) y digital-analógico (D/A), permitiendo la comunicación eficiente entre los dispositivos. Posteriormente, se realizó la caracterización de resistencias y fotorresistencias mediante un circuito de lectura, con el fin de obtener un panorama sobre los requerimientos necesarios para la caracterización de un microbolómetro real. Esta caracterización incluyó un barrido de voltaje que permitió evaluar las características I/V y la automatización de este proceso usando multiplexores. Posteriormente, se caracterizaron las fotorresistencias a diferentes condiciones de iluminación utilizando un PWM con diferentes ciclos de trabajo. También se diseñó una PCB con una matriz de 8x8 fototransistores, cuyo objetivo fue la obtención de imágenes mediante la detección de variaciones lumínicas. Finalmente, se desarrolló un sistema de adquisición de datos completo, capaz de generar imágenes a partir de los datos obtenidos de los fototransistores.
    